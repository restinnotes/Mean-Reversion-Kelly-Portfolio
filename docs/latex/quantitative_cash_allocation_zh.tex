\documentclass[12pt, a4paper]{ctexart}

% --- Basic Packages ---
\usepackage{geometry}       % Page geometry
\usepackage{amsmath, amssymb} % Math symbols
\usepackage{graphicx}       % Images
\usepackage{xcolor}         % Color support
\usepackage[skins, breakable]{tcolorbox} % Advanced text boxes
\usepackage{parskip}        % Paragraph spacing
\usepackage{booktabs}       % Professional tables
\usepackage{enumitem}       % Custom lists
\usepackage{fancyhdr}       % Headers and footers
\usepackage{lastpage}       % Reference to last page

% --- Page Setup ---
\geometry{left=2.5cm, right=2.5cm, top=3cm, bottom=3cm}
\linespread{1.4} % Slightly increased line spread for better readability

% --- Color Definitions ---
\definecolor{primaryBlue}{RGB}{0, 85, 164}   % A deep, professional blue
\definecolor{accentGreen}{RGB}{0, 153, 76}   % Green for positive concepts
\definecolor{bgGray}{RGB}{245, 247, 250}     % Very light gray background
\definecolor{textGray}{RGB}{60, 60, 60}      % Dark gray for body text

% --- Hyperlinks Setup ---
\usepackage{hyperref}
\hypersetup{
    colorlinks=true,
    linkcolor=primaryBlue,
    filecolor=magenta,
    urlcolor=primaryBlue,
    citecolor=primaryBlue,
    pdftitle={The Holy Grail of Money Management},
    pdfauthor={restinnotes} % Updated Author for PDF Metadata
}

% --- Typography & Section Styling ---
% Customize section fonts and colors
\ctexset{
    section = {
        format = \Large\bfseries\color{primaryBlue}\sffamily,
        name = {},
        number = \arabic{section}
    },
    subsection = {
        format = \large\bfseries\color{primaryBlue!80!black}\sffamily
    }
}

% --- Header & Footer ---
\pagestyle{fancy}
\fancyhf{} % Clear default
\setlength{\headheight}{15pt}
% LHead: Show Attribution
\lhead{\sffamily\small\color{primaryBlue} \href{https://github.com/restinnotes}{\textbf{GitHub}: restinnotes}}
% RHead: Show dynamic Section Title (\leftmark)
\rhead{\sffamily\small\color{textGray} \leftmark}
% LFoot: Show Attribution
\lfoot{\sffamily\small\color{gray} \href{https://github.com/restinnotes}{\textbf{GitHub}: restinnotes}}
% RFoot: Page Number
\rfoot{\sffamily\small\color{gray} Page \thepage\ of \pageref{LastPage}}
\renewcommand{\headrulewidth}{0.4pt}
\renewcommand{\footrulewidth}{0.4pt}

% --- Custom Box Environments ---

% 1. Box for Key Formulas (Enhanced)
\newtcolorbox{FormulaBox}[2][]{
    enhanced,
    colback=bgGray,
    colframe=primaryBlue,
    coltitle=white,
    fonttitle=\bfseries\sffamily,
    title={#2},
    attach boxed title to top left={yshift=-2mm, xshift=2mm},
    boxed title style={colback=primaryBlue, sharp corners},
    drop fuzzy shadow,
    #1
}

% 2. Box for Concepts/Notes
\newtcolorbox{ConceptBox}[1][]{
    enhanced,
    colback=white,
    colframe=accentGreen,
    leftrule=4mm,
    rightrule=0mm,
    toprule=0mm,
    bottomrule=0mm,
    arc=0mm,
    boxsep=1mm,
    coltext=textGray,
    #1
}

% --- List Customization ---
\setlist[itemize]{label=\textcolor{primaryBlue}{$\blacksquare$}, itemsep=2pt}
\setlist[enumerate]{label=\textcolor{primaryBlue}{\textbf{\arabic*.}}, itemsep=2pt}

% --- Document Info ---
\title{\Huge\bfseries\sffamily\color{primaryBlue} 寻找资金管理的圣杯:\\[0.5em] \Large 基于凯利公式的量化仓位管理系统}
\author{\large\sffamily \href{https://github.com/restinnotes}{\textbf{GitHub}: restinnotes}}
\date{} % Empty date

% --- Main Content ---
\begin{document}

\maketitle
\thispagestyle{empty} % No header/footer on title page
\newpage
\setcounter{page}{1}

\section{引言:寻找资金管理的圣杯}

在构建交易系统时,最难的问题往往不是“买什么”,而是\textbf{“买多少”}。
如果买得太少,资金利用率低,浪费了机会;如果买得太多,一次剧烈波动就可能导致账户归零。

本章节将详细阐述如何从最基础的数学原理出发,推导出一个量化仓位管理系统,旨在在高波动市场中最大化收益率,同时控制风险。
LEAPS 是实现杠杆与风险控制的一种工具,但核心理念在于科学地分配资金。

\section{第一阶段:起源 —— 离散凯利公式}
\textit{(The Origin: Discrete Kelly Criterion)}

一切的起点,是赌场里的抛硬币游戏。
假设你有一个胜率 $p$ 的游戏,赢了赚 $b$ 倍,输了亏光。为了让长期财富增长速度最大化(而不是单次赢最多),数学家 J.L. Kelly 推导出了著名的公式:

\begin{ConceptBox}
\textbf{基础公式:}
\[
f^* = \frac{bp - q}{b}
\]
\end{ConceptBox}

其中:
\begin{itemize}
    \item $p$: 赢的概率
    \item $q$: 输的概率 ($1-p$)
    \item $b$: 赔率
\end{itemize}

\textbf{局限性:} 这个公式假设游戏是“离散”的(玩一把就结束,要么赢要么输)。
但股票市场是“连续”的,股价每一秒都在波动,不存在“输光”或“翻倍”的绝对界限。

\section{第二阶段:进化 —— 连续时间莫顿比例}
\textit{(Evolution: Merton's Fraction)}

为了适应股市的连续波动,诺贝尔奖得主 Robert Merton 将凯利公式引入随机微积分领域。
对于一个遵循几何布朗运动的资产,最佳仓位取决于\textbf{收益率}和\textbf{波动率}的对抗。

\begin{FormulaBox}{核心公式:莫顿比例}
\[
f^* = \frac{\mu - r}{\sigma^2}
\]
\end{FormulaBox}

这个公式揭示了两个深刻的真理:
\begin{enumerate}
    \item \textbf{分子(动力):} $\mu - r$ 是超额收益(预期收益减去无风险利率)。收益越高,仓位越大。
    \item \textbf{分母(刹车):} $\sigma^2$ 是方差。注意,这里是平方!
    这意味着波动率对仓位的惩罚是指数级的。如果波动率翻倍,为了活命,你的仓位必须砍掉 75\%。
\end{enumerate}

\section{第三阶段:策略适配与杠杆实现}
\textit{(Adaptation: Strategy Adjustment and Leverage Implementation)}

在实际股市中,单纯的莫顿比例无法直接应用,因为不同标的存在估值偏离和波动率变化。
我们需要对公式进行策略性适配,将 LEAPS 视为独立资产类别,精确计算其净优势(Net Edge)。

\subsection{单标的动力与回报修正}

\textbf{1. 正股回归动力 ($\mu_{\text{stock}}$)} \\
利用均值回归原理计算正股的预期年化收益:
\[
\mu_{\text{stock}, i} = \lambda_i \cdot \ln\left(\frac{V_i}{P_i}\right)
\]
其中 $\lambda_i$ 为回归速度(弹性系数),$V_i$ 为合理估值,$P_i$ 为当前价格。偏离度越大,潜在反弹动力越强。

\textbf{2. LEAPS 净优势 ($\text{ERP}$)} \\
考虑 LEAPS 的杠杆放大效应,并严格扣除资金占用成本 ($r_f$) 与期权时间损耗 ($\theta$),得到真实的净优势:
\[
\text{ERP}_i = (\mu_{\text{stock}, i} \cdot L_i) - r_f - \theta_{\text{annual}, i}
\]
\textit{注:此处 $\theta_{\text{annual}}$ 为年化时间损耗率(正值),体现维持杠杆头寸的“租金”成本。}

\textbf{3. 估值折扣系数 ($\alpha$)} \\
引入“风险水位”概念,确保在价格接近硬底时资金利用率最高,而在接近目标价时自动收缩敞口:
\[
\alpha_i = 1 - \beta \cdot \left( \frac{P_i - P_{\text{floor}, i}}{V_i - P_{\text{floor}, i}} \right)
\]
其中 $P_{\text{floor}, i}$ 为历史极值硬底。当 $P_i \to P_{\text{floor}}$ 时,$\alpha \to 1.0$(满仓信心);当 $P_i \to V_i$ 时,$\alpha \to 1-\beta$(折扣保护)。

\subsection{LEAPS 杠杆工具说明}

为了实现非对称收益与尾部风险控制,我们使用深度实值远期期权(LEAPS):

\begin{itemize}
    \item \textbf{行权价(Strike Price)}:设置在硬底 ($P_{\text{floor}}$) 附近,物理上截断下行尾部风险。即便正股腰斩,LEAPS 的最大亏损也被锁定。
    \item \textbf{Delta 控制}:选择 $\delta > 0.8$ 的合约,获得接近正股的线性收益,提供约 2.0x - 3.0x 的有效杠杆 ($L$)。
    \item \textbf{Theta 优化}:虽然引入了时间损耗,但由于策略专门捕捉被低估标的的高 $\mu$ 值,且深度实值期权的损耗率较低,策略的高 ERP 足以覆盖 Theta 成本。
\end{itemize}

\subsection{期限选择的数学博弈:寻找完美“接盘”合约}
\textit{(The Solver: Finding the Perfect Filling Contract)}

在 LEAPS 交易中,期限 $T$ 的选择不再是凭感觉,而是为了满足一个战术目标:\textbf{确保在至暗时刻,能够打满子弹。}

我们在 Step 0.5 中引入了“最优期限求解器”,其数学本质是求解以下方程的根 $T^*$:

\[
f_{Kelly}(P=V_{fill}, \ k=k_{fill}, \ T^*) \approx 100\%
\]

\textbf{求解逻辑:}
系统遍历所有可用期限($T$),寻找一张合约,使得当股价跌至用户设定的补仓价 $V_{fill}$ 时,在应用了激进的 $k_{fill}$(通常为 1.0)后,模型建议的仓位恰好填满账户本金(100\% Allocation)。

\textbf{为什么排除 3 个月内的合约?}
求解器默认从 $T=90$ 天开始扫描。这是因为短期期权(<90天)存在两个致命缺陷:
\begin{itemize}
    \item \textbf{IV 偏高与不稳定}:短端隐含波动率往往受近期事件(如财报)扭曲,呈现“波动率微笑”极端形态,导致定价虚高,ERP 被压缩。
    \item \textbf{Gamma 风险过大}:短期期权对股价变动极其敏感,违背了 LEAPS 策略“用时间换空间”的初衷。
\end{itemize}

\textbf{结果含义:}
求解出的期限 $T^*$ 是一个“进可攻退可守”的锚点:它保证了如果市场真的发生暴跌,你手中的期权特性允许你安全地、从容地在底部完成满仓操作,而不会因为杠杆过高在半山腰就爆仓。

\section{第四阶段:终极形态 —— 单标的现金公式}
\textit{(The Final Form: Single Asset Cash Allocation)}

整合上述所有参数,我们将 LEAPS 视为一个“高波动、高收益、带成本”的合成资产,代入凯利公式得到现金投入比例:

\begin{FormulaBox}{单标的现金投入公式 (修正版)}
\[
\text{cash}_i = \text{Capital} \cdot \frac{k \cdot \max(0, \alpha_i \cdot \text{ERP}_i)}{\sigma_{\text{stock}, i}^2 \cdot L_i^2}
\]
\end{FormulaBox}

\textbf{公式核心逻辑解读:}
\begin{itemize}
    \item \textbf{分子(进攻):} $\alpha_i \cdot \text{ERP}_i$
    \begin{itemize}
        \item 代表经估值信心调整后的净优势。只有当扣除利息和时间损耗后 ERP 仍为正,才允许开仓。
    \end{itemize}
    \item \textbf{分母(防守):} $\sigma_{\text{stock}, i}^2 \cdot L_i^2$
    \begin{itemize}
        \item 这是 LEAPS 的方差 ($\text{Variance}_{\text{leaps}}$)。
        \item \textbf{稳健性修正(Robust Estimation):} 这里的 $\sigma_{\text{stock}}$ 不再使用简单的历史标准差。我们采用 \textbf{滚动窗口(Rolling Window)} 方法,计算过去 5 年中每个 252 天窗口的波动率,并取其 \textbf{85 分位数(85th Percentile)}。
        \item \textbf{安全锁(Safety Lock):} 若当前瞬时波动率高于历史 85 分位值,模型将强制使用当前值。这确保了在平静期模型不会过于激进,而在动荡期能迅速收缩战线。
        \item \textbf{杠杆惩罚}:风险随着杠杆 ($L_i$) 的\textbf{平方级}放大。必须除以 $L^2$ 才能防止爆仓。
    \end{itemize}
    \item \textbf{动态信心调节器 (Dynamic $k$):}
本系统摒弃了传统的固定 $k$ 值,引入了随股价深度 ($Depth$) 线性增强的动态 $k$ 值机制。

\[
k(P) =
\begin{cases}
k_{start} & \text{if } P \ge P_{current} \\
k_{start} + (k_{fill} - k_{start}) \cdot \frac{P_{current} - P}{P_{current} - V_{fill}} & \text{if } V_{fill} < P < P_{current} \\
k_{fill} & \text{if } P \le V_{fill}
\end{cases}
\]

\textbf{逻辑含义}:
\begin{itemize}
    \item \textbf{建仓期 ($P_{current}$)}:使用保守的 $k_{start}$ (如 0.5)。此时安全边际尚未最大化,主要目标是建立底仓并控制回撤。
    \item \textbf{击球区 ($V_{fill}$)}:当股价跌至预设的补仓价时,均值回归的势能积蓄至顶点,此时 $k$ 线性增加至 $k_{fill}$ (如 1.0),指示投资者在胜率最高的时刻敢于重仓出击。
\end{itemize}
\end{itemize}

\section{第六阶段:多标的仓位处理 —— 归一化 vs 矩阵凯利}

在实际投资中,我们可能同时持有很多标的。针对如何分配资金,我们提供两种由浅入深的思路。

\subsection{方法一:简单归一化 (Simple Normalization)}
\textit{适用于:小组合、标的间相关性低、或追求计算简便的场景。}

假设有 $N$ 个标的,根据单标的公式计算出名义现金投入 $\text{cash}_i$。

\textbf{1. 计算原始总仓位占用:}
\[
C_{\text{raw}} = \sum_{i=1}^{N} \text{cash}_i
\]

\textbf{2. 设定最大杠杆上限 ($C_{\max}$):}
例如,设定账户总资金占用不超过 100\% 或 120\%。

\textbf{3. 线性缩放:}
若 $C_{\text{raw}} > C_{\max}$,则对所有标的进行等比例压缩:
\[
\text{cash}_i^{\text{final}} = \text{cash}_i \cdot \frac{C_{\max}}{C_{\text{raw}}}
\]
若 $C_{\text{raw}} \le C_{\max}$,则保持原仓位不变。

\textbf{评价:}
\begin{itemize}
    \item \textbf{优点}:直观,永远不会爆仓,保留了各标的之间 ERP 的相对强弱关系。
    \item \textbf{缺点}:忽略了相关性。如果持仓全是高度相关的半导体股(如 NVDA, AMD, TSM),即使归一化后,组合的实际波动风险依然可能过大。
\end{itemize}

\subsection{方法二:多维矩阵凯利 (Multivariate Matrix Kelly)}
\textit{适用于:大资金、标的关联度高、追求数学上最优解的场景。}

当标的之间存在相关性时,单标的公式 ($\mu/\sigma^2$) 不再适用,必须引入\textbf{协方差矩阵}。

\textbf{1. 定义参数:}
\begin{itemize}
    \item $\mathbf{E}$: 超额收益向量 ($N \times 1$),其中第 $i$ 个元素为 $\alpha_i \cdot \text{ERP}_i$。
    \item $\mathbf{\Sigma}$: LEAPS 收益率的协方差矩阵 ($N \times N$)。注意,这里的元素应为 $\text{Cov}(r_i \cdot L_i, r_j \cdot L_j)$。
\end{itemize}

\textbf{2. 矩阵凯利公式:}
最优仓位向量 $\mathbf{F}^*$(每个元素代表该标的占总资金的比例)由下式给出:

\begin{FormulaBox}{矩阵凯利优化公式}
\[
\mathbf{F}^* = k \cdot \mathbf{\Sigma}^{-1} \cdot \mathbf{E}
\]
\end{FormulaBox}

其中 $\mathbf{\Sigma}^{-1}$ 是协方差矩阵的逆矩阵。

\textbf{3. 数学含义:}
\begin{itemize}
    \item 如果两个标的 $i, j$ 高度正相关(同涨同跌),$\mathbf{\Sigma}^{-1}$ 会自动降低两者的仓位权重,起到“惩罚拥挤交易”的作用。
    \item 如果两个标的负相关(对冲),公式会自动放大仓位,因为组合风险被对冲降低了。
\end{itemize}

\textbf{评价:}
\begin{itemize}
    \item \textbf{优点}:这是理论上的“圣杯”,能精准处理相关性风险,实现风险调整后收益最大化。
    \item \textbf{缺点}:计算复杂,且对协方差矩阵的估计误差非常敏感(垃圾进,垃圾出)。通常需要对结果施加非负约束($\text{cash}_i \ge 0$)和总杠杆约束。
\end{itemize}

\subsection{实战建议}

\begin{enumerate}
    \item \textbf{默认推荐:} 使用 \textbf{方法一(归一化)}。虽然粗糙,但配合我们保守的 $L^2$ 分母设置,通常已经足够安全。
    \item \textbf{进阶风控:} 如果持仓标的超过 5 只且处于同一板块,建议人为降低全组合的 $k$ 值(例如从 1.0 降至 0.7),以模拟相关性带来的风险。
\end{enumerate}

\end{document}