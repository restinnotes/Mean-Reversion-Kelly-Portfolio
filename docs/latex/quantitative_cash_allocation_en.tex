\documentclass[12pt, a4paper]{article}

% --- Basic Packages ---
\usepackage{geometry}         % Page margins
\usepackage{amsmath, amssymb} % Math formulas
\usepackage{graphicx}         % Images
\usepackage{xcolor}           % Colors
\usepackage{hyperref}         % Hyperlinks
\usepackage[skins]{tcolorbox} % Text boxes (skins loaded for 'enhanced')
\usepackage{parskip}          % Paragraph spacing
\usepackage{booktabs}         % Table lines

% --- Page Settings ---
\geometry{left=2.5cm, right=2.5cm, top=2.5cm, bottom=2.5cm}
\linespread{1.3} % Line spacing

% --- Color Definitions ---
\definecolor{myblue}{RGB}{0,102,204}
\definecolor{mygreen}{RGB}{0,153,76}

% --- tcolorbox Settings ---
\tcbset{enhanced, colback=myblue!5, colframe=myblue}

\title{\textbf{From Kelly Criterion to Quantitative Position Management: \\ A Capital Management System for Maximizing Returns}}
\author{Quantitative Strategy Documentation}

\begin{document}

\date{}
\maketitle

\section{Introduction: The Holy Grail of Capital Management}

In building a trading system, the hardest question is often not ``what to buy,'' but \textbf{``how much to buy.''}
If you buy too little, capital utilization is low, and opportunities are wasted. If you buy too much, a single violent fluctuation can wipe out the account.

This document details the derivation of a quantitative position management system starting from fundamental mathematical principles. It aims to maximize returns in high-volatility markets while strictly controlling risk.
While LEAPS are the tool used for leverage and risk control, the core philosophy lies in the scientific allocation of capital.

\section{Phase 1: The Origin — Discrete Kelly Criterion}

Everything starts with a coin-tossing game in a casino.
Suppose you have a game with a winning probability $p$. If you win, you gain $b$ times your bet; if you lose, you lose the bet. To maximize the long-term growth rate of wealth (rather than maximizing the win of a single round), mathematician J.L. Kelly derived the famous formula:

\[
f^* = \frac{bp - q}{b}
\]

Where:
\begin{itemize}
    \item $p$: Probability of winning
    \item $q$: Probability of losing ($1-p$)
    \item $b$: Odds (Payout ratio)
\end{itemize}

\textbf{Limitation:} This formula assumes the game is ``discrete'' (one round ends with a binary outcome).
However, the stock market is ``continuous.'' Prices fluctuate every second, and there is no absolute boundary of ``losing everything'' or ``doubling'' in a single instance.

\section{Phase 2: Evolution — Continuous Time Merton Fraction}

To adapt to the continuous fluctuations of the stock market, Nobel Laureate Robert Merton introduced the Kelly Criterion into the field of stochastic calculus.
For an asset following Geometric Brownian Motion, the optimal position depends on the battle between \textbf{Return} and \textbf{Volatility}.

Core Formula (Merton's Fraction):
\[
f^* = \frac{\mu - r}{\sigma^2}
\]

This formula reveals two profound truths:
\begin{enumerate}
    \item \textbf{Numerator (Engine):} $\mu - r$ is the Excess Return (Expected Return minus Risk-Free Rate). The higher the return, the larger the position.
    \item \textbf{Denominator (Brake):} $\sigma^2$ is the Variance. Note that this is squared!
          This means the penalty for volatility on position size is exponential. If volatility doubles, to survive, your position size must be cut by 75\%.
\end{enumerate}

\section{Phase 3: Strategy Adaptation \& Leverage Implementation}

In the actual stock market, the pure Merton Fraction cannot be applied directly because different assets have varying valuation deviations and volatility changes.
We need to adapt the formula strategically, treating LEAPS as an independent asset class and calculating its Net Edge precisely.

\subsection{Single Asset Momentum \& Return Correction}

\textbf{1. Stock Mean Reversion Force ($\mu_{\text{stock}}$)} \\
Using the principle of mean reversion, we calculate the expected annualized return of the underlying stock:
\[
\mu_{\text{stock}, i} = \lambda_i \cdot \ln\left(\frac{V_i}{P_i}\right)
\]
Where $\lambda_i$ is the reversion speed (elasticity coefficient), $V_i$ is the fair value, and $P_i$ is the current price. The greater the deviation, the stronger the potential rebound momentum.

\textbf{2. LEAPS Net Edge ($\text{ERP}$)} \\
Considering the leverage amplification effect of LEAPS, and strictly deducting the cost of capital ($r_f$) and option time decay ($\theta$), we get the true Net Edge:
\[
\text{ERP}_i = (\mu_{\text{stock}, i} \cdot L_i) - r_f - \theta_{\text{annual}, i}
\]
\textit{Note: Here, $\theta_{\text{annual}}$ is the annualized time decay rate (positive value), representing the ``rent'' cost of maintaining the leveraged position.}

\textbf{3. Valuation Discount Coefficient ($\alpha$)} \\
Introducing the concept of ``Risk Water Level'' ensures maximum capital utilization when prices are near the ``Hard Floor,'' and automatically contracts exposure when approaching the Target Price:
\[
\alpha_i = 1 - \beta \cdot \left( \frac{P_i - P_{\text{floor}, i}}{V_i - P_{\text{floor}, i}} \right)
\]
Where $P_{\text{floor}, i}$ is the historical valuation floor. When $P_i \to P_{\text{floor}}$, $\alpha \to 1.0$ (Full Confidence); when $P_i \to V_i$, $\alpha \to 1-\beta$ (Discount Protection).

\subsection{LEAPS Leverage Tool Specification}

To achieve asymmetric returns and tail risk control, we use Deep In-The-Money (ITM) LEAPS:

\begin{itemize}
    \item \textbf{Strike Price:} Set near the Hard Floor ($P_{\text{floor}}$) to physically truncate downside tail risk. Even if the stock halves, the maximum loss of the LEAPS is capped.
    \item \textbf{Delta Control:} Select contracts with $\delta > 0.8$ to obtain linear returns close to the underlying stock, providing an effective leverage ($L$) of approximately 2.0x - 3.0x.
    \item \textbf{Theta Optimization:} Although time decay is introduced, since the strategy targets undervalued assets with high $\mu$ values, and Deep ITM options have lower decay rates, the strategy's high ERP is sufficient to cover Theta costs.
\end{itemize}

\section{Phase 4: The Final Form — Single Asset Cash Formula}

Integrating all the parameters above, we treat LEAPS as a ``high volatility, high return, cost-bearing'' synthetic asset. Substituting into the Kelly formula yields the cash allocation ratio:

\begin{tcolorbox}[title={V23.1 Single Asset Cash Allocation Formula (Revised)}]
\[
\text{cash}_i = \text{Capital} \cdot \frac{k \cdot \max(0, \alpha_i \cdot \text{ERP}_i)}{\sigma_{\text{stock}, i}^2 \cdot L_i^2}
\]
\end{tcolorbox}

\textbf{Core Logic Interpretation:}
\begin{itemize}
    \item \textbf{Numerator (Offense):} $\alpha_i \cdot \text{ERP}_i$
    \begin{itemize}
        \item Represents the Net Edge adjusted for valuation confidence. A position is opened only if the ERP remains positive after deducting interest and time decay.
    \end{itemize}
    \item \textbf{Denominator (Defense):} $\sigma_{\text{stock}, i}^2 \cdot L_i^2$
    \begin{itemize}
        \item This represents the Variance of the LEAPS ($\text{Variance}_{\text{leaps}}$).
        \item \textbf{Critical Correction:} Risk scales with the \textbf{square} of the leverage ($L_i$). It must be divided by $L^2$ to prevent blowing up under high leverage. This is the mathematical iron law of applying Kelly to derivatives.
    \end{itemize}
    \item \textbf{Regulator:} $k$ is the Kelly fractional factor (e.g., 0.5 or 1.0), used to control overall aggressiveness.
\end{itemize}

\section{Phase 6: Portfolio Management — Normalization vs. Matrix Kelly}

In actual investing, we may hold multiple assets simultaneously. We provide two approaches for capital allocation, ranging from simple to advanced.

\subsection{Method 1: Simple Normalization}
\textit{Suitable for: Small portfolios, low correlation between assets, or scenarios favoring calculation simplicity.}

Assume there are $N$ assets. Calculate the nominal cash allocation $\text{cash}_i$ for each asset using the single asset formula.

\textbf{1. Calculate Raw Total Exposure:}
\[
C_{\text{raw}} = \sum_{i=1}^{N} \text{cash}_i
\]

\textbf{2. Set Maximum Leverage Cap ($C_{\max}$):}
For example, set the total account capital usage not to exceed 100\% or 120\%.

\textbf{3. Linear Scaling:}
If $C_{\text{raw}} > C_{\max}$, compress all positions proportionally:
\[
\text{cash}_i^{\text{final}} = \text{cash}_i \cdot \frac{C_{\max}}{C_{\text{raw}}}
\]
If $C_{\text{raw}} \le C_{\max}$, keep original positions unchanged.

\textbf{Evaluation:}
\begin{itemize}
    \item \textbf{Pros:} Intuitive, prevents blowing up, preserves the relative strength of ERP among assets.
    \item \textbf{Cons:} Ignores correlation. If the portfolio consists entirely of highly correlated semiconductor stocks (e.g., NVDA, AMD, TSM), the actual portfolio volatility risk may still be too high even after normalization.
\end{itemize}

\subsection{Method 2: Multivariate Matrix Kelly}
\textit{Suitable for: Large capital, high asset correlation, or scenarios seeking the theoretically optimal mathematical solution.}

When correlation exists between assets, the single asset formula ($\mu/\sigma^2$) no longer applies, and a \textbf{Covariance Matrix} must be introduced.

\textbf{1. Define Parameters:}
\begin{itemize}
    \item $\mathbf{E}$: Excess Return Vector ($N \times 1$), where the $i$-th element is $\alpha_i \cdot \text{ERP}_i$.
    \item $\mathbf{\Sigma}$: LEAPS Return Covariance Matrix ($N \times N$). Note that elements here should be $\text{Cov}(r_i \cdot L_i, r_j \cdot L_j)$.
\end{itemize}

\textbf{2. Matrix Kelly Formula:}
The optimal position vector $\mathbf{F}^*$ (each element represents the fraction of total capital for that asset) is given by:

\begin{tcolorbox}[colback=white, colframe=myblue]
\[
\mathbf{F}^* = k \cdot \mathbf{\Sigma}^{-1} \cdot \mathbf{E}
\]
\end{tcolorbox}

Where $\mathbf{\Sigma}^{-1}$ is the inverse of the covariance matrix.

\textbf{3. Mathematical Meaning:}
\begin{itemize}
    \item If two assets $i, j$ are highly positively correlated (move together), $\mathbf{\Sigma}^{-1}$ will automatically reduce the weights of both, acting as a ``penalty for crowded trades.''
    \item If two assets are negatively correlated (hedging), the formula will automatically increase positions because the portfolio risk is reduced via hedging.
\end{itemize}

\textbf{Evaluation:}
\begin{itemize}
    \item \textbf{Pros:} This is the theoretical ``Holy Grail,'' capable of precisely handling correlation risk and maximizing risk-adjusted returns.
    \item \textbf{Cons:} Computationally complex and extremely sensitive to estimation errors in the covariance matrix (``Garbage In, Garbage Out''). It typically requires imposing non-negative constraints ($\text{cash}_i \ge 0$) and total leverage constraints.
\end{itemize}

\subsection{Practical Recommendations}

\begin{enumerate}
    \item \textbf{Default Recommendation:} Use \textbf{Method 1 (Normalization)}. Although coarse, combined with our conservative $L^2$ denominator setting, it is usually safe enough.
    \item \textbf{Advanced Risk Control:} If holding more than 5 assets in the same sector, it is recommended to manually lower the full portfolio $k$ value (e.g., from 1.0 to 0.7) to simulate the risk brought by correlation.
\end{enumerate}

\end{document}