\documentclass[12pt, a4paper]{ctexart}

% --- 基础包配置 ---
\usepackage{geometry}         % 页面边距
\usepackage{amsmath, amssymb} % 数学公式
\usepackage{graphicx}         % 图片插入
\usepackage{xcolor}           % 颜色
\usepackage{hyperref}         % 超链接
\usepackage[skins]{tcolorbox} % 漂亮的文本框 (加载 skins 以支持 enhanced)
\usepackage{parskip}          % 段落间距
\usepackage{booktabs}         % 表格线

% --- 页面设置 ---
\geometry{left=2.5cm, right=2.5cm, top=2.5cm, bottom=2.5cm}
\linespread{1.3} % 行间距

% --- 颜色定义 ---
\definecolor{myblue}{RGB}{0,102,204}
\definecolor{mygreen}{RGB}{0,153,76}

% --- tcolorbox 设置 ---
\tcbset{enhanced, colback=myblue!5, colframe=myblue}

\title{\textbf{从凯利公式到量化仓位管理:\\一个最大化收益率的资金管理系统}}
\author{量化策略文档}

\begin{document}

\date{}
\maketitle

\section{引言:寻找资金管理的圣杯}

在构建交易系统时,最难的问题往往不是“买什么”,而是\textbf{“买多少”}。
如果买得太少,资金利用率低,浪费了机会;如果买得太多,一次剧烈波动就可能导致账户归零。

本章节将详细阐述如何从最基础的数学原理出发,推导出一个量化仓位管理系统,旨在在高波动市场中最大化收益率,同时控制风险。
LEAPS 是实现杠杆与风险控制的一种工具,但核心理念在于科学地分配资金。

\section{第一阶段:起源 —— 离散凯利公式}
\textit{(The Origin: Discrete Kelly Criterion)}

一切的起点,是赌场里的抛硬币游戏。
假设你有一个胜率 $p$ 的游戏,赢了赚 $b$ 倍,输了亏光。为了让长期财富增长速度最大化(而不是单次赢最多),数学家 J.L. Kelly 推导出了著名的公式:

\[
f^* = \frac{bp - q}{b}
\]

其中:
\begin{itemize}
    \item $p$: 赢的概率
    \item $q$: 输的概率 ($1-p$)
    \item $b$: 赔率
\end{itemize}

\textbf{局限性:} 这个公式假设游戏是“离散”的(玩一把就结束,要么赢要么输)。
但股票市场是“连续”的,股价每一秒都在波动,不存在“输光”或“翻倍”的绝对界限。

\section{第二阶段:进化 —— 连续时间莫顿比例}
\textit{(Evolution: Merton's Fraction)}

为了适应股市的连续波动,诺贝尔奖得主 Robert Merton 将凯利公式引入随机微积分领域。
对于一个遵循几何布朗运动的资产,最佳仓位取决于\textbf{收益率}和\textbf{波动率}的对抗。

核心公式(莫顿比例):
\[
f^* = \frac{\mu - r}{\sigma^2}
\]

这个公式揭示了两个深刻的真理:
\begin{enumerate}
    \item \textbf{分子(动力):} $\mu - r$ 是超额收益(预期收益减去无风险利率)。收益越高,仓位越大。
    \item \textbf{分母(刹车):} $\sigma^2$ 是方差。注意,这里是平方!
          这意味着波动率对仓位的惩罚是指数级的。如果波动率翻倍,为了活命,你的仓位必须砍掉 75\%。
\end{enumerate}

\section{第三阶段:策略适配与杠杆实现}
\textit{(Adaptation: Strategy Adjustment and Leverage Implementation)}

在实际股市中,单纯的莫顿比例无法直接应用,因为不同标的存在估值偏离和波动率变化。
我们需要对公式进行策略性适配,将 LEAPS 视为独立资产类别,精确计算其净优势(Net Edge)。

\subsection{单标的动力与回报修正}

\textbf{1. 正股回归动力 ($\mu_{\text{stock}}$)} \\
利用均值回归原理计算正股的预期年化收益:
\[
\mu_{\text{stock}, i} = \lambda_i \cdot \ln\left(\frac{V_i}{P_i}\right)
\]
其中 $\lambda_i$ 为回归速度(弹性系数),$V_i$ 为合理估值,$P_i$ 为当前价格。偏离度越大,潜在反弹动力越强。

\textbf{2. LEAPS 净优势 ($\text{ERP}$)} \\
考虑 LEAPS 的杠杆放大效应,并严格扣除资金占用成本 ($r_f$) 与期权时间损耗 ($\theta$),得到真实的净优势:
\[
\text{ERP}_i = (\mu_{\text{stock}, i} \cdot L_i) - r_f - \theta_{\text{annual}, i}
\]
\textit{注:此处 $\theta_{\text{annual}}$ 为年化时间损耗率(正值),体现维持杠杆头寸的“租金”成本。}

\textbf{3. 估值折扣系数 ($\alpha$)} \\
引入“风险水位”概念,确保在价格接近硬底时资金利用率最高,而在接近目标价时自动收缩敞口:
\[
\alpha_i = 1 - \beta \cdot \left( \frac{P_i - P_{\text{floor}, i}}{V_i - P_{\text{floor}, i}} \right)
\]
其中 $P_{\text{floor}, i}$ 为历史极值硬底。当 $P_i \to P_{\text{floor}}$ 时,$\alpha \to 1.0$(满仓信心);当 $P_i \to V_i$ 时,$\alpha \to 1-\beta$(折扣保护)。

\subsection{LEAPS 杠杆工具说明}

为了实现非对称收益与尾部风险控制,我们使用深度实值远期期权(LEAPS):

\begin{itemize}
    \item \textbf{行权价(Strike Price)}:设置在硬底 ($P_{\text{floor}}$) 附近,物理上截断下行尾部风险。即便正股腰斩,LEAPS 的最大亏损也被锁定。
    \item \textbf{Delta 控制}:选择 $\delta > 0.8$ 的合约,获得接近正股的线性收益,提供约 2.0x - 3.0x 的有效杠杆 ($L$)。
    \item \textbf{Theta 优化}:虽然引入了时间损耗,但由于策略专门捕捉被低估标的的高 $\mu$ 值,且深度实值期权的损耗率较低,策略的高 ERP 足以覆盖 Theta 成本。
\end{itemize}

\section{第四阶段:终极形态 —— 单标的现金公式}
\textit{(The Final Form: Single Asset Cash Allocation)}

整合上述所有参数,我们将 LEAPS 视为一个“高波动、高收益、带成本”的合成资产,代入凯利公式得到现金投入比例:

\begin{tcolorbox}[title={单标的现金投入公式 (修正版)}]
\[
\text{cash}_i = \text{Capital} \cdot \frac{k \cdot \max(0, \alpha_i \cdot \text{ERP}_i)}{\sigma_{\text{stock}, i}^2 \cdot L_i^2}
\]
\end{tcolorbox}

\textbf{公式核心逻辑解读:}
\begin{itemize}
    \item \textbf{分子(进攻):} $\alpha_i \cdot \text{ERP}_i$
    \begin{itemize}
        \item 代表经估值信心调整后的净优势。只有当扣除利息和时间损耗后 ERP 仍为正,才允许开仓。
    \end{itemize}
    \item \textbf{分母(防守):} $\sigma_{\text{stock}, i}^2 \cdot L_i^2$
    \begin{itemize}
        \item 这是 LEAPS 的方差 ($\text{Variance}_{\text{leaps}}$)。
        \item \textbf{关键修正}:风险是随着杠杆 ($L_i$) 的\textbf{平方级}放大的。必须除以 $L^2$ 才能防止在高杠杆下爆仓,这是凯利公式应用于衍生品的数学铁律。
    \end{itemize}
    \item \textbf{调节器:} $k$ 为凯利部分因子(如 0.5 或 1.0),用于控制整体激进程度。
\end{itemize}

\section{第六阶段:多标的仓位处理 —— 归一化 vs 矩阵凯利}

在实际投资中,我们可能同时持有很多标的。针对如何分配资金,我们提供两种由浅入深的思路。

\subsection{方法一:简单归一化 (Simple Normalization)}
\textit{适用于:小组合、标的间相关性低、或追求计算简便的场景。}

假设有 $N$ 个标的,根据单标的公式计算出名义现金投入 $\text{cash}_i$。

\textbf{1. 计算原始总仓位占用:}
\[
C_{\text{raw}} = \sum_{i=1}^{N} \text{cash}_i
\]

\textbf{2. 设定最大杠杆上限 ($C_{\max}$):}
例如,设定账户总资金占用不超过 100\% 或 120\%。

\textbf{3. 线性缩放:}
若 $C_{\text{raw}} > C_{\max}$,则对所有标的进行等比例压缩:
\[
\text{cash}_i^{\text{final}} = \text{cash}_i \cdot \frac{C_{\max}}{C_{\text{raw}}}
\]
若 $C_{\text{raw}} \le C_{\max}$,则保持原仓位不变。

\textbf{评价:}
\begin{itemize}
    \item \textbf{优点}:直观,永远不会爆仓,保留了各标的之间 ERP 的相对强弱关系。
    \item \textbf{缺点}:忽略了相关性。如果持仓全是高度相关的半导体股(如 NVDA, AMD, TSM),即使归一化后,组合的实际波动风险依然可能过大。
\end{itemize}

\subsection{方法二:多维矩阵凯利 (Multivariate Matrix Kelly)}
\textit{适用于:大资金、标的关联度高、追求数学上最优解的场景。}

当标的之间存在相关性时,单标的公式 ($\mu/\sigma^2$) 不再适用,必须引入\textbf{协方差矩阵}。

\textbf{1. 定义参数:}
\begin{itemize}
    \item $\mathbf{E}$: 超额收益向量 ($N \times 1$),其中第 $i$ 个元素为 $\alpha_i \cdot \text{ERP}_i$。
    \item $\mathbf{\Sigma}$: LEAPS 收益率的协方差矩阵 ($N \times N$)。注意,这里的元素应为 $\text{Cov}(r_i \cdot L_i, r_j \cdot L_j)$。
\end{itemize}

\textbf{2. 矩阵凯利公式:}
最优仓位向量 $\mathbf{F}^*$(每个元素代表该标的占总资金的比例)由下式给出:

\begin{tcolorbox}[colback=white, colframe=myblue]
\[
\mathbf{F}^* = k \cdot \mathbf{\Sigma}^{-1} \cdot \mathbf{E}
\]
\end{tcolorbox}

其中 $\mathbf{\Sigma}^{-1}$ 是协方差矩阵的逆矩阵。

\textbf{3. 数学含义:}
\begin{itemize}
    \item 如果两个标的 $i, j$ 高度正相关(同涨同跌),$\mathbf{\Sigma}^{-1}$ 会自动降低两者的仓位权重,起到“惩罚拥挤交易”的作用。
    \item 如果两个标的负相关(对冲),公式会自动放大仓位,因为组合风险被对冲降低了。
\end{itemize}

\textbf{评价:}
\begin{itemize}
    \item \textbf{优点}:这是理论上的“圣杯”,能精准处理相关性风险,实现风险调整后收益最大化。
    \item \textbf{缺点}:计算复杂,且对协方差矩阵的估计误差非常敏感(垃圾进,垃圾出)。通常需要对结果施加非负约束($\text{cash}_i \ge 0$)和总杠杆约束。
\end{itemize}

\subsection{实战建议}

\begin{enumerate}
    \item \textbf{默认推荐:} 使用 \textbf{方法一(归一化)}。虽然粗糙,但配合我们保守的 $L^2$ 分母设置,通常已经足够安全。
    \item \textbf{进阶风控:} 如果持仓标的超过 5 只且处于同一板块,建议人为降低全组合的 $k$ 值(例如从 1.0 降至 0.7),以模拟相关性带来的风险。
\end{enumerate}

\end{document}